\chapter*{ABSTRACT}
\label{tomtat}

A large proportion of our perceptions of the world around us is made up of opinions.
In large scale, opinions run our economy, shape our government and drive our history.
The ability to comprehend and control opinions brings great power.
Since the last decade, opinions have been expressed in the highest speed, volume, and complexity ever recorded in history.
As a result, the task of tracking the flow of opinions on the Internet become impossible without automation.
To fill the niche, Sentiment Analysis is the field of study that investigates the computational method for analyzing opinions.
In this field, one significant problem is to classify the sentiment expressed by a sentence.
This problem is named Sentence-level Sentiment Analysis.

The purpose of this thesis is to increase the classification accuracy on the task of sentence-level Sentiment Analysis.
For this purpose, three approaches were explored.
In the first approach, we tried to improve Recursive Neural Networks by parameterizing their composition functions with respect to local syntactic information at each node in the parse tree.
For our second approach, we recognized that one major obstacle of this task is the lack of context and knowledge in the small training set.
To tackle this problem, we utilized Glove method to do Transfer Learning on a large amount of document-level labeled sentiment data.
In the third approach, we combined Convolution Neural Networks with Recursive Neural Networks, which result in new sophisticated networks architect.
To mitigate the risk posed by training large networks on small training dataset, we experimented with several unsupervised pre-training methods which also types of Transfer Learning.

We evaluated our models on the public Stanford Sentiment Treebank dataset with binary setting.
We were able to outperform the state-of-the-art model by a good margin in terms average accuracy of 5 runs.
