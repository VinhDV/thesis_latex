\chapter{Introduction}
\section{Context and purpose of this thesis}
Before a customer buys any product, the decision of whether or not he will buy it depends largely on his prior opinions about that product. 
These opinions in turn had been build based on his opinions on related companies or products and other customers' opinions about that product.  
After had been experiencing the product, his posterior opinions on it not only telling us about which features he like or being dissatisfied with.
They also inform us about the reasons why he had bought the product, his expectations, needs and even more detail information about him.
In a circle, his opinions will also affect the opinion of new customers and even the design of future products.
As a result, customers' opinions are the controller behind every companies' good decisions. 
They shapes companies' marketing strategies, policies, designs of products.
They decide which company is more competent than another company.

Long time ago, when companies needed to know opinions of their customers, they conducted surveys, opinion pull and focus groups\cite{liu2012sentiment}. 
In recent years, thanks to the dramatic growth of social media, customers' opinions have been expressed in the highest speed and volume ever recorded in history.
With this amount of data, it is inefficient to read and analyze or even gather them manually.
To deal with this problem, Sentiment Analysis is the field of study that computationally analyze opinions, sentiments, evaluations, appraisals, attitudes, and emotions being expressed\cite{liu2012sentiment}.
By it definition, Sentiment Analysis not only applicable to customer reviews, in near future, we may also find it application in management sciences, political science, economics, and social sciences as they are all largely affected by opinions\cite{liu2012sentiment}. 


\begin{itemize}
\item The important role of sentiment analysis in academic as well as business
\item The importance and application of sentence level sentiment analysis
\item The rise of deep learning and it application in NLP and the task of sentiment analysis
\item The challenge faced by Deep Learning when dealing with sentiment analysis
\item Which challenge this thesis will tackle
\end{itemize}

\section{Contributions of this thesis}
\begin{itemize}
\item Analysis and discussion of other recent approaches to sentiment analysis for movie reviews 
\item Improve dependency tree of Kai Seng Tai
\item Demonstrate the effectiveness of several method of transfer-learning and unsupervised pre-training  
\end{itemize}

\section{Structure of this thesis}
\begin{itemize}
\item Chapter 2
\item Chapter 3
\item Chapter 4
\item Chapter 5
\item Chapter 6
\end{itemize}